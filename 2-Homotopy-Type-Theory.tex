\chapter{Homotopy Type Theory}

\section{Homotopical Language}

\begin{defn}
    A \textsl{pointed type} $(A,a)$ is a type $A\colon\univ$ together with a point $a\colon A$, called its \textsl{basepoint}. We write
    \[
        \univ_\sbull\defeq\sum_{A:\,\univ}A
    \]
    for the type of pointed types in the universe $\univ$.
\end{defn}

\begin{defn}
    Given a pointed type $(A,a)$, we define the \textsl{loop space} of $(A,a)$ to be the following pointed type:
    \[
        \Omega(A,a)\defeq((a\eq Aa),\fun{refl}_a).
    \]
    An element of it will be called a \textsl{loop} at $a$. For $n\colon\N$, the \textsl{$n$-fold iterated loop space} $\Omega^n(A,a)$ of a pointed type $(A,a)$ is defined recursively by:
    \begin{align*}
        \Omega^0(A,a)&\defeq(A,a)\\
        \Omega^{n+1}(A,a)&\defeq\Omega^n(\Omega(A,a)).
    \end{align*}
    An element of it will be called an \textsl{$n$-loop} or an \textsl{$n$dimensional loop} at $a$.

    Note that $\Omega^1(A,a)\equiv\Omega(A,a)$.
\end{defn}

\begin{rem}
    Recall from Exercise~\ref{exr:ap-application} that given $f\colon A\to B$, the congruence function
    \[
        \fun{ap}_f\colon\prod_{x,y\colon A}\;\prod_{p\colon x\eq Ay}f(x)\eq Bf(y)
    \]
    transforms paths in $A$ to paths in $B$. Specifically
    \[
        \fun{ap}_f(x,y,p)\colon f(x)\eq Bf(y),\quad\text{for }p\colon x\eq Ay,
    \]
    where the notation $\fun{ap}_f(x,y,p)$ is usually simplified to $\fun{ap}_f(p)$ or $\tilde f(p)$.
\if{false}    
    Moreover,
    \[
        \fun{ap}_f(\fun{refl}_x)\equiv\fun{refl}_{f(x)}
        \quad\text{and}\quad
        \fun{ap}_f(p\ct q)\eq {f(x)\eq Bf(z)}
            \fun{ap}_f(p)\ct \fun{ap}_f(q).
    \]
\fi
\end{rem}

\begin{lem}
    For functions\/ $f\colon A\to B$ and\/ $g\colon B\to C$ and paths\/ $p\colon x\eq Ay$ and\/ $q\colon y\eq Az$, we have:
    \begin{enumerate}[a),font=\upshape]
        \item $\tilde f(\fun{refl}_x)\equiv\fun{refl}_{f(x)}$.
        \item $\tilde f(p\ct q)\eq{f(x)\eq Bf(z)}\tilde f(p)\ct\tilde f(q)$.
        \item $\tilde f(p^{-1})\eq{f(y)\eq Bf(x)}\tilde f(p)^{-1}$.
        \item $\tilde g(\tilde f(p))\eq{h(x)\eq Ch(y)}\tilde h(p)$, where $h\defeq g\circ f$.
        \item If $B\equiv A$ and $f\equiv\id_A$, then $\widetilde{\id}_A(p)\eq{x\eq Ay}p$.
    \end{enumerate}
\end{lem}

\begin{proof}${}$
    \begin{enumerate}[a)]
        \item Exercise~\ref{exr:ap-application}.
        \item Exercise~\ref{exr:ap-application}.
        
        \item First observe that
        \begin{align*}
            \tilde f(p)\ct\tilde f(p^{-1})
                &\eq{f(x)\eq Bf(x)}\tilde f(p\ct p^{-1})
                    &&\text{; by part b)}\\
                &\eq{f(x)\eq Bf(x)}\tilde f(\fun{refl}_x)
                    &&\text{; inverse law}\\
                &\equiv\fun{refl}_{f(x)}
                    &&\text{; by part a).}
        \end{align*}
        Therefore,
        \begin{align*}
            \tilde f(p)^{-1}
                &\eq{f(y)\eq Bf(x)}
                    \tilde f(p)^{-1}\ct\big(\tilde f(p) \ct\tilde f(p^{-1})\big)
                    &&\text{; right unit}\\
                &\eq{f(y)\eq Bf(x)}
                    \big(\tilde f(p)^{-1}\ct\tilde f(p)\big) \ct\tilde f(p^{-1})
                    &&\text{; associativity}\\
                &\eq{f(y)\eq Bf(x)}\tilde f(p^{-1})
                    &&\text{; left inverse.}
        \end{align*}

        \item Since $p\colon x\eq Ay$, we know that $\tilde f(p)\colon f(x)\eq Bf(y)$. Therefore,
        \[
            \tilde g(\tilde f(p))\colon g(f(x))\eq Cg(f(y)).
        \]
        Let $h\defeq g\circ f$. Note that the type of the term above is definitionally equal to $h(x)\eq C h(y)$.
        Consider the case where $p\equiv\fun{refl}_x$. We have
        \begin{align*}
            \tilde g(\tilde f(\fun{refl}_x))
                &\equiv \tilde g(\fun{refl}_{f(x)})\\
                &\equiv \fun{refl}_{g(f(x))}\\
                &\equiv \fun{refl}_{h(x)}\\
                &\equiv \tilde h(\fun{refl}_x).
        \end{align*}
        To conclude that $\tilde g(\tilde f(p))\eq{h(x)\eq Ch(y)}\tilde h(p)$, we apply path induction to the motive
        \[
            C(x,y,p)\defeq \tilde g(\tilde f(p))\eq{h(x)\eq Ch(y)}\tilde h(p),
        \]
        with the base case $c(x)\colon C(x,x,\fun{refl}_x)$ defined as $\fun{refl}_{\tilde h(\fun{refl}_x)}$.

        \item For\/ $p\equiv\fun{refl}_x$, since\/ $\id_A(x)\equiv x$, we have
        \begin{align*}
            \widetilde{\id}_A(\fun{refl}_x)
                &\equiv \fun{refl}_{\id_A(x)}\\
                &\equiv \fun{refl}_x.
        \end{align*}
        Therefore, the conclusion follows by path induction applied to the motive
        \[
            C(x,y,p)\defeq\widetilde{\id}_A(p)\eq{x\eq Ay}p
        \]
        with base case\/ $c(x)\defeq\fun{refl}_{\fun{refl}_x}\colon C(x,x,\fun{refl}_x)$.
        
    \end{enumerate}
\end{proof}

\begin{rem}
    Recall from classical topology that a \textsl{fibration} is a continuous map\/ $\pi\colon E\to B$ with the homotopy-lifting property:
    \[
        \begin{tikzcd}[row sep=large, column sep=huge]
            X
                \arrow[r,"f"]
                \arrow[d,hook,"i_0"']
            &E
                \arrow[d,"\pi"] \\
            X\times I
                \arrow[r,"H"']
                \arrow[ur,dashed,"\tilde{H}"]
            &B
        \end{tikzcd}
    \]
    In this context, $E$ is the \textsl{total space} and\/ $B$ the \textsl{base space} of the fibration.

    \textsl{Serre fibrations} lift paths, path homotopies, and higher-dimensional volumes. For instance, in the case of paths, the diagram above becomes:
    \[
        \begin{tikzcd}[row sep=large, column sep=large]
            \{0\}
                \arrow[r, "e_0"]
                \arrow[d, hook, "i"']
            & E
                \arrow[d, "\pi"] \\
            I
                \arrow[r, "\gamma"']
                \arrow[ur, dashed, "\tilde{\gamma}"]
            & B
        \end{tikzcd}
    \]
    In Homotopy Type Theory, given a type family\/ $P\colon A\to\univ$, the first projection\/ $\fun{\pi}_1\colon \sum_{x\colon A}P(x)\to A$ is analogous to a fibration with \textsl{total space}\/ $\sum_{x\colon A}P(x)$ and \textsl{base space}\/ $A$. This structural analogy is established by the following lemma.
\end{rem}

\begin{lem}{\upshape[Path Lifting Property]}
    Let\/ $P\colon A\to\univ$ be a type family over\/ $A$ and let\/ $u\colon P(a)$ for some\/ $a\colon A$. Then, for any path\/ $p\colon a\eq Ax$, there is a path
    \[
        \fun{lift}(u,p)\colon(a,u)\eq{\sum_{y\colon A}P(y)}
            (x,\fun{transport}^P(p,u))
    \]
    such that
    \[
        \fun{ap}_{\fun{\pi}_1}(\fun{lift}(u, p))\eq{a\eq Ax}p,
    \]
    where\/ $\pi_1\colon\sum_{y\colon A}P(y)\to A$ is the first projection.
\end{lem}

\begin{proof}
    Consider the motive family\/ $C\colon\prod_{x\colon A}(a\eq Ax)\to\univ$, defined by
    \[
        C(x,p)\defeq(a,u)\eq{\sum_{y\colon A}P(y)}(x,\fun{transport}^P(p,u)).
    \]
    Since\/ $\fun{transport}^P(\fun{refl}_a,u)\equiv u$, we have
    \[
        C(a,\fun{refl}_a)\equiv
            (a,u)\eq{\sum_{y\colon A}P(y)}(a,u),
    \]
    and we can specify the base case\/ $c\defeq\fun{refl}_{(a,u)}$ to define, by \nref{lpar:based-path-induction},
    \[
        \fun{lift}(u,p)\defeq\fun{ind}'_{\eq A}(a,C,c,x,p).
    \]
    We can use based path induction again to verify that
    \[
        \fun{ap}_{\fun{\pi}_1}(\fun{lift}(u,p)) \eq{a\eq Ax} p.
    \]
    For the base case, we know definitionally that\/ $\fun{lift}(u,\fun{refl}_a) \equiv \fun{refl}_{(a,u)}$. Thus,
    \[
        \fun{ap}_{\fun{\pi}_1}(\fun{lift}(u,\fun{refl}_a))
        \equiv \fun{ap}_{\fun{\pi}_1}(\fun{refl}_{(a,u)})
        \equiv \fun{refl}_{\pi_1(a,u)}
        \equiv \fun{refl}_a.
    \]
\end{proof}

\begin{rem}\label{rem:section-type}
    The lemma can be represented by the following diagram, where the horizontal arrows represent paths ---i.e., terms of the propositional equalities between their ends--- and the vertical arrows, functions:
    \[
        \begin{tikzcd}[row sep=large, column sep=huge]
            (a,u)
                \arrow[r,"{\fun{lift}(u,p)}"]
                \arrow[d,"\pi_1"']
            &(x, \fun{transport}^P(p,u))
                \arrow[d,"\pi_1"] \\
            a
                \arrow[r,"p"']
            &x
        \end{tikzcd}
        \qquad;\ u\colon P(a),\;p\colon a\eq Ax.
    \]
    Note also that a term $f\colon\prod_{x\colon A}P(x)$ can be regarded as a \textsl{section} of the fibration induced by~$P$.
\end{rem}

\begin{ntn}
    When\/ $P$ is clear from the context we will simply write
    \[
        p_*(u)\defeq\fun{transport}^P(p,u)\colon P(x).
    \]
    In particular,
    \begin{equation}\label{eq:refl-*}
        (\fun{refl}_a)_*(u)\equiv u.
    \end{equation}
\end{ntn}

\begin{lem}{\upshape[Dependent map]}
    Let\/ $P\colon A\to\univ$. If\/ $f\colon\prod_{x\colon A}P(x)$, then there is a map
    \[
        \fun{apd}_f\colon\prod_{x,y\colon A}\;
            \prod_{p\colon x\eq Ay}
                p_*(f(x))\eq{P(y)}f(y)
    \]
    satisfying $\fun{apd}_f(\fun{refl}_x)\equiv\fun{refl}_{f(x)}$.
\end{lem}

\begin{proof}
    Fix $a\colon A$ and consider the motive
    \begin{align*}
        C&\colon\prod_{a\colon A}(a\eq Ax)\to\univ\\
        C(x,p)&\defeq p_*(f(a))\eq{P(x)}f(x).
    \end{align*}
    Since
    \begin{align*}
        C(a,\fun{refl}_a)
            &\equiv (\fun{refl}_a)_*(f(a))\eq{P(a)}f(a)\\
            &\equiv f(a)\eq{P(a)}f(a)
                &&\text{; \eqref{eq:refl-*}},
    \end{align*}
    we can specify $c\defeq\fun{refl}_{f(a)}$ and obtain $\fun{apd}_f$ by based path induction.
\end{proof}

\begin{lem}
    If\/ $P \colon A \to \univ$ is defined by\/ $P(x) :\equiv B$ for a fixed\/ $B \colon \univ$, then for any\/ $x, y \colon A$ and\/ $p \colon x \eq A y$ and\/ $b \colon B$ we have a path
    \[
        \fun{transportconst}^B_p(b) \colon \fun{transport}^P(p, b) \eq B b
    \]
    satisfying\/ $\fun{transportconst}^B_{\fun{refl}_x}\defeq\lambda b.\,\fun{refl}_b$.
\end{lem}

\begin{proof}
    See Exercise \ref{exr:constant-fibration}.
\end{proof}

\begin{lem}
    Let\/ $f \colon A \to B$ and define the constant family\/ $P \colon A \to \univ$ by\/ $P(x) :\equiv B$. For any path\/ $p \colon x \eq A y$, we have
    \[
        \fun{apd}_f(p)\eq{p_*(f(x))\eq B f(y)}
            \fun{transportconst}^B_p(f(x))\ct\fun{ap}_f(p).
    \]
\end{lem}

\begin{proof}
    Fix\/ $a\colon A$ and let\/ $C\colon\prod_{x\colon A}(a\eq Ax)\to\univ$ be defined by
    \[
        C(x,p)\defeq\fun{apd}_f(p)\eq{p_*(f(a))\eq Bf(x)}
            \fun{transportconst}^B_p(f(a))\ct\fun{ap}_f(p).
    \]
    To verify the base case:
    \begin{align*}
        C(a,\fun{refl}_a)
            &\equiv\fun{apd}_f(\fun{refl}_a)
                \eq{(\fun{refl}_a)_*(f(a))\eq Bf(a)}
                \fun{transportconst}^B_{\fun{refl}_a}(f(a))
                \ct\fun{ap}_f(\fun{refl}_a)\\
            &\equiv\fun{refl}_{f(a)}
                \eq{f(a)\eq Bf(a)}
                \fun{refl}_{f(a)}\ct\fun{refl}_{f(a)}\\
            &\equiv \fun{refl}_{f(a)}
                \eq{f(a)\eq Bf(a)}
                \fun{refl}_{f(a)}.
    \end{align*}
    Therefore, we can specify the base case\/ $c\defeq\fun{refl}_{\fun{refl}_{f(a)}}$ to complete the proof by based path induction.
\end{proof}

\begin{rem}
    We can represent the previous lemma in a propositionally commutative path diagram, where ``composition'' means concatenation:
    \[
        \begin{tikzcd}[column sep=2.8cm,row sep=1.5cm]
            p_*(f(x))
                    \arrow[r,"\fun{transportconst}^B_p(f(x))"]
                    \arrow[rd,"\fun{apd}_f(p)"']&f(x)
                    \arrow[d,"\fun{ap}_f(p)"]\\
                &f(y)
        \end{tikzcd}
    \]    
\end{rem}

\begin{lem}
    Given\/ $P\colon A\to\univ$, $p\colon x\eq Ay$ and, $q\colon y \eq Az$, and\/ $u\colon P(x)$, we have
    \[
        q_*(p_*(u))\eq{P(z)}(p\ct q)_*(u).
    \]
\end{lem}

\begin{proof}
    Fix $x\colon A$. We proceed by based path induction on $p\colon x\eq A y$. The motive of the induction is the family of types $D\colon \prod_{y\colon A} (x\eq A y)\to \univ$ defined by
    \[
        D(y, p) \defeq \prod_{u\colon P(x)}\; \prod_{z\colon A}\; \prod_{q\colon y\eq A z}
            q_*(p_*(u)) \eq{P(z)} (p\ct q)_*(u).
    \]
    We must construct a term of type $D(x, \fun{refl}_x)$. Substituting $y$ with $x$ and $p$ with $\fun{refl}_x$, the goal becomes
    \[
        \prod_{u\colon P(x)}\;
            \prod_{z\colon A}\;
            \prod_{q\colon x\eq Az} 
                q_*((\fun{refl}_x)_*(u))\eq{P(z)}
                    (\fun{refl}_x\ct q)_*(u).
    \]
    Using the definitional equalities $(\fun{refl}_x)_*(u) \equiv u$ and $\fun{refl}_x\ct q \equiv q$, this reduces to
    \[
        \prod_{u\colon P(x)}\;
            \prod_{z\colon A}\;
                \prod_{q\colon x\eq Az}
                    q_*(u)\eq{P(z)}q_*(u),
    \]
    which is inhabited by $\lambda u,z,q.\,\fun{refl}_{q_*(u)}$.
\end{proof}

\begin{lem}
    For a function\/ $f \colon A \to B$, a type family\/ $P \colon B \to \univ$, and any\/ $p \colon x \eq A y$ and\/ $u \colon P(f(x))$, we have
    \[
        \fun{transport}^{P\circ f}(p,u) \eq{P(f(y))}
            \fun{transport}^P(\fun{ap}_f(p),u).
    \]
\end{lem}

\begin{proof}
    We proceed by path induction on $p$. Define the motive
    \[
        C\colon \prod_{y\colon A} (x\eq A y)\to \univ
    \]
    by
    \[
        C(y,p) \defeq
            \prod_{u\colon P(f(x))}
                \fun{transport}^{P\circ f}(p,u) \eq{P(f(y))}
                    \fun{transport}^P(\fun{ap}_f(p),u).
    \]
    Evaluating at the base case, we have
    \begin{align*}
        C(x,\fun{refl}_x)
            &\equiv \prod_{u\colon P(f(x))}
                \fun{transport}^{P\circ f}(\fun{refl}_x,u)
                    \eq{P(f(x))}
                    \fun{transport}^P(\fun{ap}_f(\fun{refl}_x),u) \\
            &\equiv \prod_{u\colon P(f(x))}
                u \eq{P(f(x))}
                    \fun{transport}^P(\fun{refl}_{f(x)},u) \\
            &\equiv \prod_{u\colon P(f(x))}
                u \eq{P(f(x))} u,
    \end{align*}
    which is inhabited by $\lambda u\colon P(f(x)).\,\fun{refl}_u$.
\end{proof}

\begin{lem}
    For\/ $P, Q \colon A \to \univ$, a function family\/ $f \colon \prod_{x\colon A} P(x) \to Q(x)$, and any\/ $p \colon x \eq A y$ and\/ $u \colon P(x)$, we have
    \[
        \fun{transport}^Q(p,f_x(u))\eq{Q(y)}
            f_y(\fun{transport}^P(p,u)).
    \]
\end{lem}

\begin{proof}
    To use based path induction on $p$, fix $a\colon A$ and consider the motive 
    \begin{align*}
        C&\colon\prod_{x\colon A}(a\eq Ax)\to\univ\\
        C(x,p)&\defeq\prod_{u\colon P(a)}
            \fun{transport}^Q(p,f_a(u))\eq{Q(x)}
                f_x(\fun{transport}^P(p,u)).
    \end{align*}
    For the base case we obtain
    \begin{align*}
        C(a,\fun{refl}_a)
            &\equiv\prod_{u\colon P(a)}
                \fun{transport}^Q(\fun{refl}_a,f_a(u))\eq{Q(a)}
                    f_a(\fun{transport}^P(\fun{refl}_a,u))\\
            &\equiv\prod_{u\colon P(a)}f_a(u)\eq{Q(a)}f_a(u),
    \end{align*}
    which is inhabited by $\lambda u\colon P(a).\,\fun{refl}_{f_a(u)}$.
\end{proof}

%
%
\begin{defn}
    Given a function\/ $f\colon A\to B$ and a point\/ $a\colon A$, the \textsl{induced map on loops} is the function
    \[
        \Omega f \colon \Omega(A, a) \to \Omega(B, f(a))
    \]
    defined by
    \[
        \Omega f(p) \defeq \fun{ap}_f(p).
    \]
    Note that by part a) of the previous lemma, we have\/ $\Omega f(\fun{refl}_a) \equiv \fun{refl}_{f(a)}$, meaning that\/ $\Omega f$ preserves the basepoint definitionally.
\end{defn}

\[
    \begin{tikzcd}[column sep=huge,row sep=large]
        a
                \arrow[r,bend left=45,"p"{name=U1}]
                \arrow[r,bend right=45,"q"'{name=D1}]
            &b
                \arrow[r,bend left=45,"r"{name=U2}]
                \arrow[r,bend right=45,"s"'{name=D2}]
            &c
                \arrow[Rightarrow,from=U1,to=D1,"\alpha",
                    shorten <= 8pt,shorten >= 8pt]
                \arrow[Rightarrow,from=U2,to=D2,"\beta",
                    shorten <= 8pt,shorten >= 8pt]
    \end{tikzcd}
\]

\begin{thm} {\upshape[Eckmann-Hilton]}
    The composition operation on the second loop space
    \[
        \Omega^2(A)\times\Omega^2(A)\to\Omega^2(A)
    \]
    is commutative: $\alpha\ct\beta=\beta\ct\alpha$, for any\/ $\alpha,\beta\colon\Omega^2(A)$.
\end{thm}

\section{Function Homotopies}

\begin{ntn}
    Given a type family $P\colon A\to\univ$, we will sometimes denote type of sections\footnote{Remark \ref{rem:section-type}.} $\prod_{x\colon A}P(x)$ by $\Gamma(P)$.
\end{ntn}

\begin{defn}
    Given two sections\/ $f,g\colon\prod_{x\colon A}P(x)$ of a family\/ $P\colon A\to\univ$, a \textsl{homotopy} from\/ $f$ to\/ $g$ is a dependent function of type
    \[
        f\sim g\defeq\prod_{x\colon A}f(x)\eq\univ g(x).
    \]
\end{defn}

\begin{prop}
    % Define a length to store the width of the longest subscript.
    \newlength{\maxsubwidth}
    % Measure the widest subscript (the third one) in scriptstyle (standard for limits).
    \settowidth{\maxsubwidth}{$\scriptstyle f,g,h\colon\Gamma(P)$}
    % Define a helper command to wrap the subscript in a centered box of that fixed width.
    \newcommand{\alignedprod}[1]{\prod_{\makebox[\maxsubwidth][c]{$\scriptstyle #1$}}}
    Homotopy is an equivalence relation on each dependent function type\/ $\prod_{x\colon A}P(x)$. That is, we have elements of the types
    \begin{align*}
        &\alignedprod{f\colon\Gamma(P)}
            f\sim f\\
        &\alignedprod{f,g\colon\Gamma(P)}
            (f\sim g)\to(g\sim f)\\
        &\alignedprod{f,g,h\colon\Gamma(P)}
            (f\sim g)\to(g\sim h)\to(f\sim h).
    \end{align*}
\end{prop}

\begin{proof}
    We have the following inhabitants for each of the\/ $\Pi$-types:
    \begin{itemize}
        \item \textit{Reflexivity:}\/ $\lambda f\colon\Gamma(P).\,\lambda x\colon A.\,\fun{refl}_{f(x)}$.
        \item \textit{Symmetry:}\/ $\lambda f,g\colon\Gamma(P).\,\lambda \eta\colon f\sim g.\,\lambda x\colon A.\,\eta(x)^{-1}$.
        \item \textit{Transitivity:}\/ $\lambda f,g,h\colon\Gamma(P).\,\lambda\eta\colon f\sim g.\,\lambda\vartheta\colon g\sim h.\,\lambda x\colon A.\,\eta(x)\ct\vartheta(x)$.
    \end{itemize}
\end{proof}

\begin{rem}\label{rem:types-as-categories} {\upshape[Types as categories]}
    Types admit different interpretations; depending on the circumstances we can see them as sets, propositions, spaces. Another way to think of types is as categories. In this analogy we have
    \begin{description}[font=\normalfont\scshape]
        \item[Objects.] These are the elements of the type: $x,y\colon A$ represent the objects that populate the category $A$.

        \item[Arrows.] Given two objects $x,y\colon A$, an arrow is a path $p\colon x\eq Ay$.

        \item[Composition.] Composition is concatenation. Note however that if $p\colon x\eq Ay$ and $q\colon y\eq Az$, then $p\ct q\colon x\eq Az$ is denoted in the reverse order when compared to categorical composition. We could also consider $q\circ p\defeq p\ct q$. Note that concatenation is propositionally associative.

        \item[Identity.] Given $x\colon A$ the identity for concatenation is $\fun{refl}_x$. Note that $\fun{refl}_x$ is the propositional unit of concatenation. 

        \item[Functors.] A function $f\colon A\to B$ acts as a functor between the corresponding categories:
        \begin{center}
            \begin{tabular}{lll}
                \small
                \textbf{Feature}
                    &\textbf{Domain}
                    &\textbf{Codomain}\\
                \hline\rule{0mm}{3.5mm}\small
                Objects
                    &$x\colon A$
                    &$f(x)\colon B$\\[0.7mm]
                \small
                Morphisms
                    &$p\colon x\eq A y$
                    &$\tilde f(p)\colon f(x)\eq Bf(y)$\\[0.7mm]
                \small
                Composition
                    &$p\ct q$
                    &$\tilde f(p)\ct\tilde f(q)$
            \end{tabular}
        \end{center}

        \item[Natural transformations.] These are the homotopies: given $f,g\colon A\to B$, a natural transformation between them is an inhabitant of the homotopy type $\eta\colon f\sim g$. To see this we need to show the (propositional) commutativity of the square
        \begin{equation}\label{eq:nat-transformation}
            \begin{tikzcd}
                f(x)
                        \arrow[r,"\tilde f(p)"]
                        \arrow[d,"\eta_x"']
                    &f(y)
                        \arrow[d,"\eta_y"]\\
                g(x)
                        \arrow[r,"\tilde g(p)"']
                    &g(y)
            \end{tikzcd}
        \end{equation}
        i.e., for every $p\colon x\eq Ay$, the type
        \begin{align*}
            \tilde f(p)\ct\eta_y\eq B\eta_x\ct\tilde g(p)
        \end{align*}
        is inhabited. This follows by based path induction on $p$ using the motive $C\colon\prod_{y\colon A}(x\eq Ay)\to\univ$ defined by
        \[
            C(y,p)\defeq
                \tilde f(p)\ct\eta_y\eq B\eta_x\ct\tilde g(p),
        \]
        which satisfies
        \begin{align*}
            C(x,\fun{refl}_x)
                &\defeq\tilde f(\fun{refl}_x)\ct\eta_x
                    \eq B\eta_x\ct\tilde g(p)\\
                &\equiv\fun{refl}_{f(x)}\ct\eta_x
                    \eq B\eta_x\ct\fun{refl}_{g(x)}\\
                &\equiv \eta_x\eq B\eta_x\ct\fun{refl}_{g(x)}.
        \end{align*}
        By Lemma~\ref{lem:right-unit-law}, $\fun{unit}_r(\eta_x)\colon\eta_x\ct\fun{refl}_{g(x)}\eq B\eta_x$. Hence, $\fun{unit}_r(\eta_x)^{-1}$ is an inhabitant $C(x,\fun{refl}_x)$.\footnote{To achieve independence of our definition of concatenation, we could also have used the left unit law $\fun{unit}_l$, that satisfies $\fun{unit}_l(p)\colon\fun{refl}_x\ct p\eq Ap$, arriving at $\fun{unit}_l(\eta_x)\ct\fun{unit}_r(\eta_x)^{-1}$.}
    \end{description}
\end{rem}

\begin{lem}\label{lem:tilde-id-is-id}
    For any\/ $x,y\colon A$ and\/ $p\colon x\eq A y$, we have
    \[
        \fun{ap}_{\id_A}(p) = p.
    \]
\end{lem}

\begin{proof}
    By path induction, it suffices to assume\/ $p$ is\/ $\fun{refl}_x$.
    The left-hand side reduces to\/ $\fun{ap}_{\id_A}(\fun{refl}_x)$, which is defined as\/ $\fun{refl}_{\id_A(x)} \equiv \fun{refl}_x$.
    The right-hand side is\/ $\fun{refl}_x$.
    Thus, the equality holds by\/ $\fun{refl}_{\fun{refl}_x}$.
\end{proof}

\begin{prop}
    Let\/ $\eta\colon f\sim\id_A$ be a homotopy, with\/ $f\colon A\to A$. Then for any\/ $x\colon A$ we have
    \[
        \eta_{f(x)}\eq A\tilde f(\eta_x).
    \]
\end{prop}

\begin{proof}
    By \eqref{eq:nat-transformation}, we have a commutative path diagram
    \[
        \begin{tikzcd}
        f(f(x))
                \arrow[r,"\tilde f(\eta_x)"]
                \arrow[d,"\eta_{f(x)}"']
            &f(x)
                \arrow[d,"\eta_x"]\\
        f(x)
                \arrow[r,"\eta_x"']
            &x
        \end{tikzcd}
    \]
    where, according to Lemma~\ref{lem:tilde-id-is-id}, we have replaced $\widetilde\id_A(\eta_x)$ with $\eta_x$. The conclusion follows from concatenation with $\eta_x^{-1}$.
\end{proof}

\begin{lem}{\upshape[Whiskering]}\label{lem:whiskering}
    The homotopy relation is compatible with composition:
    \begin{description}[font=\normalfont\scshape]
        \item[Right:] Given\/ $\eta\colon f\sim g$ and a right composable function\/ $h$, we have
        \[
            \eta\cdot h\defeq\lambda z.\,\eta(h(z))\colon f\circ h\sim g\circ h.
        \]
        \item[Left:] Given a left composable function\/ $h$ and\/ $\eta\colon f\sim g$, we have
        \[
            h\cdot\eta\defeq\lambda x.\,\fun{ap}_h(\eta(x))\colon h\circ f\sim h\circ g.
        \]
    \end{description}
\end{lem}

\begin{proof}
    Assume that\/ $f,g\colon A\to B$ and let\/ $\eta\colon f\sim g$ be a homotopy.

    First, consider the case where\/ $h\colon C\to A$. Given\/ $z\colon C$, we have\/ $h(z)\colon A$ and therefore a path\/ $\eta(h(z))\colon f(h(z))\eq B g(h(z))$. Hence, the term\/ $\lambda z.\,\eta(h(z))$ is a well-defined witness of\/ $f\circ h\sim g\circ h$.

    For the case where\/ $h\colon B\to C$, the function\/ $\fun{ap}_h$ constructs a witness of the equality\/ $h(f(x))\eq C h(g(x))$ from the witness\/ $\eta(x)$ of\/ $f(x)\eq B g(x)$. Thus, the term\/ $\lambda x.\,\fun{ap}_h(\eta(x))$ is a well-defined witness of\/ $h\circ f\sim h\circ g$.
\end{proof}

\section{Equivalences}

\begin{defns}${}$
    \begin{itemize}
        \item A function $f\colon A\to B$ is an \textsl{equivalence}, when the type
        \[
            \fun{isequiv}(f) :\equiv
                \Bigl(\sum_{g\colon B\to A}f\circ g \sim \id_B\Bigr)
                \times
                \Bigl(\sum_{h\colon B\to A}h\circ f\sim\id_A\Bigr)
        \]
        is inhabited.
    
        \item A \textsl{quasi-inverse} of a function\/ $f \colon A \to B$ is a is a triple\/ $(g,\eta,\varepsilon)$ consisting of a function\/ $g\colon B\to A$ and homotopies\/ $\eta\colon f\circ g\sim \id_B$ and\/ $\varepsilon\colon g\circ f\sim \id_A$. More precisely, a quasi-inverse is an inhabitant of the type
        \[
            \fun{qinv}(f)\defeq\sum_{g\colon B\to A}(f\circ g\sim\id_B)
                \times(g\circ f\sim\id_A)
        \]
    \end{itemize}
\end{defns}

\begin{prop}
Let\/ $f\colon A\to B$. Then,
  \begin{enumerate}[a),font=\upshape]
    \item There is a map\/ $\fun{qinv}(f)\to\fun{isequiv}(f)$.
    \item There is a map\/ $\fun{isequiv}(f)\to\fun{qinv}(f)$.
  \end{enumerate}
\end{prop}

\begin{proof}${}$
  \begin{enumerate}[a),font=\upshape]
    \item This is immediate: map the triple $(g,\eta,\vartheta)$ to the tuple $(g,\eta,g,\vartheta)$.

    \item Let $(g,\eta,h,\vartheta)\colon\fun{isequiv}(f)$. By definition, we have homotopies:
    \[
      \eta\colon f\circ g\sim\id_B
      \quad\text{and}\quad
      \vartheta\colon h\circ f\sim\id_A.
    \]
    By Lemma~\ref{lem:whiskering}, we obtain:
    \begin{align*}
      h\cdot\eta &\colon h\circ f\circ g\sim h, \\
      \vartheta\cdot g &\colon h\circ f\circ g\sim g.
    \end{align*}
    Thus, we can define $\theta\colon g\sim h$, as follows:
    \[
        \theta\defeq\lambda y.\,
            \vartheta(g(y))^{-1}\ct\fun{ap}_h(\eta(y)).
    \]
    In particular, when $y\equiv f(x)$, we have
    \[
      \theta(f(x))\colon g(f(x))\eq A h(f(x)).
    \]
    Concatenating this path with $\vartheta(x)$ yields
    \[
      \theta(f(x))\ct\vartheta(x)\colon g(f(x))\eq A x.
    \]
    Thus, the triple $(g,\eta,\vartheta')$ inhabits $\fun{qinv}(f)$, where
    \[
      \vartheta'\defeq\lambda x.\,\theta(f(x))\ct\vartheta(x).
    \]
  \end{enumerate}
\end{proof}

\begin{defn}
    An \textsl{equivalence} from $A$ to $B$ is a function $f\colon A\to B$ together with an inhabitant of $\fun{isequiv}(f)$, i.e., a proof that it is an equivalence. We write $A\simeq B$ for the type of equivalences from $A$ to $B$, i.e., the type
    \[
        A\simeq B\defeq\!\!\!\sum_{f\colon A\to B}\fun{isequiv}(f).
    \]
\end{defn}

\begin{prop}
    Type equivalence of types is an equivalence relation on\/ $\univ$. More specifically:
    \begin{enumerate}[a),font=\upshape]
        \item For any\/ $A$, the identity function\/ $\id_A$ is an equivalence; hence\/ $A\simeq A$.
        
        \item For any\/ $f\colon A\simeq B$, we have an equivalence\/ $f^{-1}\colon B\simeq A$.
        
        \item For any\/ $f\colon A\simeq B$ and\/ $g\colon B\simeq C$, we have\/ $g\circ f\colon A\simeq C$.
    \end{enumerate}
\end{prop}

\begin{proof}${}$
    \begin{enumerate}[a)]
        \item Let\/ $\fun{refl} \defeq \lambda x.\,\fun{refl}_x$. Since\/ $\id_A \circ \id_A \equiv \id_A$, we clearly have a homotopy\/ $\fun{refl}\colon \id_A\circ\id_A\sim\id_A$.
        Therefore,
        \[
          (\id_A,\fun{refl},\id_A,\fun{refl})\colon\fun{isequiv}(\id_A).
        \]
        Hence,
        \[
          \big(\id_A, (\id_A,\fun{refl},\id_A,\fun{refl}))
            \colon A\simeq A.
        \]

        \item Suppose that $(g,\eta,h,\vartheta)\colon\fun{isequiv}(f)$. Then
    \end{enumerate}
    
\end{proof}